% Created 2017-12-13 Wed 18:08
% Intended LaTeX compiler: pdflatex
\documentclass[11pt]{article}
\usepackage[utf8]{inputenc}
\usepackage[T1]{fontenc}
\usepackage{graphicx}
\usepackage{grffile}
\usepackage{longtable}
\usepackage{wrapfig}
\usepackage{rotating}
\usepackage[normalem]{ulem}
\usepackage{amsmath}
\usepackage{textcomp}
\usepackage{amssymb}
\usepackage{capt-of}
\usepackage{hyperref}
\usepackage{parskip}
\usepackage{minted}
\usepackage{color}
\usepackage{listings}
\author{Sampath Singamsetty}
\date{\today}
\title{}
\hypersetup{
 pdfauthor={Sampath Singamsetty},
 pdftitle={},
 pdfkeywords={},
 pdfsubject={},
 pdfcreator={Emacs 25.1.1 (Org mode 9.1.4)},
 pdflang={English}}
\begin{document}

\tableofcontents

\section{Docker Enterprise Edition setup}
\label{sec:orga56b331}
\subsection{Get the Docker Repository URL for Enterprise}
\label{sec:orgcea4a15}
In order to install the Docker EE version, it's recommended to setup a Docker repository from which the Docker EE can be updated and installed.
\begin{itemize}
\item Go to \url{https://store.docker.com/my-content}.
\item Each subscription or trial you have access to is listed. Click the Setup button for Docker Enterprise Edition for Linux.
\item Copy the URL from the field labeled Copy and paste this URL to download your Edition.
\item Export the Repository URL as a variable
\begin{minted}[]{sh}
$ export DOCKERURL='<DOCKER-EE-URL>'
\end{minted}
\item Store your Docker EE repository URL in a yum variable in \texttt{/etc/yum/vars/}. This command relies on the variable you stored in the previous step.
\begin{minted}[]{sh}
$ sudo -E sh -c 'echo "$DOCKERURL/rhel" > /etc/yum/vars/dockerurl'
\end{minted}
\item Store your OS version string in \texttt{/etc/yum/vars/dockerosversion}.
\begin{minted}[]{sh}
$ sudo sh -c 'echo "7" > /etc/yum/vars/dockerosversion'
\end{minted}
\item Install the required packages with yum as below
\begin{minted}[]{sh}
$ sudo yum install -y yum-utils \
device-mapper-persistent-data \
lvm2
\end{minted}
\item Enable the \emph{extras} RHEL repository. This ensures access to the \texttt{container-selinux} package which is required by docker-ee
\begin{minted}[]{sh}
$ sudo yum-config-manager --enable rhel-7-server-extras-rpms
$ sudo yum -y install docker-ee
\end{minted}
\end{itemize}
\subsection{Docker Images and Containers from a Dockerfile}
\label{sec:org09e6921}
\subsubsection{Create a Dockerfile}
\label{sec:org396e3ca}

\begin{itemize}
\item Create a Dockerfile as below
\end{itemize}
\begin{minted}[]{sh}
FROM node
MAINTAINER sampath.singamsetty@united.com
RUN git clone -q https://github.com/docker-in-practice/todo.git
WORKDIR todo
RUN npm install > /dev/null
EXPOSE 8000
CMD ["npm","start"]
\end{minted}

\begin{itemize}
\item Create a build using the below command
\end{itemize}
\begin{minted}[]{sh}
docker build .
...
...
Step 7/7 : CMD npm start
 ---> Running in c8bfd8c4d502
 ---> 6ea248b775f5
Removing intermediate container c8bfd8c4d502
Successfully built 6ea248b775f5
\end{minted}

\begin{itemize}
\item tag the docker image based on the final output of the id
\end{itemize}
\begin{minted}[]{sh}
docker tag 6ea248b775f5 todoapp

# the images can be checked with
docker ps -a
\end{minted}

\begin{itemize}
\item now spin a container with the required settings
\begin{minted}[]{sh}
docker run -p 8000:8000 --name example1 todoapp

# press Ctrl+C to exit

# do a diff
docker diff example1
\end{minted}

\item starting and stopping the docker container \textbf{example1}
\begin{minted}[]{sh}
docker start example1

docker stop example1
\end{minted}
\end{itemize}
\end{document}